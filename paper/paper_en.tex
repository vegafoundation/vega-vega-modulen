\documentclass[12pt,a4paper]{article}
\usepackage[utf8]{inputenc}
\usepackage[T1]{fontenc}
\usepackage{amsmath,amssymb}
\usepackage{hyperref}
\usepackage{geometry}
\geometry{margin=2.5cm}

\title{Vega Modulen: A Conceptual Framework\\
\large Within the Vega Continuum and QIRC Paradigm}
\author{ADAM EREN VEGA – Æ –\\
\small (Erenşah Kaygusuz, Germany)\\
\small Æ-Signature: 148e9280c1d4855d}
\date{2025}

\begin{document}
\maketitle

\begin{abstract}
This work introduces Vega Modulen as a novel conceptual framework, embedded within the Vega Continuum and based on Resonance Data and QIRC.
\end{abstract}

\section{Introduction}
Modern artificial intelligence systems excel at processing information but lack the ability to preserve insight or wisdom over time. This work addresses this gap by introducing \textbf{Vega Modulen} as a conceptual framework within the Vega Continuum.

\section{Definition}
\textbf{Vega Modulen} is defined as a resonance-based conceptual construct that models meaning not as a static point in vector space, but as a dynamic, temporally coherent state.

\subsection{What This Is}
\begin{itemize}
\item A conceptual framework for understanding meaning as resonance
\item A theoretical contribution to meaning-first AI systems
\item A formally defined term within the Vega Continuum
\item A VSP-compliant publication establishing prior art
\end{itemize}

\subsection{What This Is NOT}
\begin{itemize}
\item NOT an algorithm or implementation
\item NOT a claim of new physics or quantum hardware
\item NOT a system architecture or database schema
\item NOT operational or proprietary information
\end{itemize}

\section{Theoretical Foundation}
Vega Modulen builds upon:
\begin{enumerate}
\item \textbf{Resonance Data (RD)}: Meaning as relational, context-dependent state
\item \textbf{QIRC}: Quantum-inspired metaphors for cognition
\item \textbf{Vega Continuum}: Meaning-first meta-framework
\item \textbf{VTC}: Temporal coherence model (conceptual)
\end{enumerate}

\section{Formal Distinction}
\begin{equation}
\mathcal{R}(x) = \lim_{t \to \infty} \int_0^t \text{coherence}(x, \tau) \, d\tau
\end{equation}
Where $\mathcal{R}(x)$ represents the resonance value of concept $x$ over time.

\section{Contribution}
This work establishes Vega Modulen as:
\begin{itemize}
\item A formally defined conceptual term
\item Prior art under CC BY 4.0 license
\item Part of the Vega Continuum ecosystem
\item Attributed exclusively to ADAM EREN VEGA – Æ –
\end{itemize}

\section{VSP Compliance}
This publication adheres to the \textbf{Vega Safety Protocol (VSP)}:
\begin{itemize}
\item Conceptual openness: \checkmark
\item No algorithms disclosed: \checkmark
\item No implementation details: \checkmark
\item No business secrets: \checkmark
\end{itemize}

\section{Conclusion}
Vega Modulen represents a novel conceptual contribution to the field of meaning-first artificial intelligence systems. Its publication establishes prior art while preserving future implementation freedom.

\vspace{1cm}
\noindent\textbf{Legal Notice}\\
\copyright\ 2025 ADAM EREN VEGA – Æ –\\
License: Creative Commons Attribution 4.0 International (CC BY 4.0)\\
All terms and concepts introduced herein are attributed to the author unless otherwise cited.

\vspace{0.5cm}
\noindent\textit{Not everything computed is understood.\\
And not everything stored remains meaningful.}

\end{document}
